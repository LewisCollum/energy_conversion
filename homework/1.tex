% Created 2019-10-29 Tue 19:02
% Intended LaTeX compiler: pdflatex
\documentclass[11pt]{article}
\usepackage[utf8]{inputenc}
\usepackage[T1]{fontenc}
\usepackage{graphicx}
\usepackage{grffile}
\usepackage{longtable}
\usepackage{wrapfig}
\usepackage{rotating}
\usepackage[normalem]{ulem}
\usepackage{amsmath}
\usepackage{textcomp}
\usepackage{amssymb}
\usepackage{capt-of}
\usepackage{hyperref}
\usepackage{minted}
\usepackage{homework}
\usepackage{mathcomp}
\author{Lewis Collum}
\date{Updated: \today}
\title{EE331 Fall 2019 HW \jobname}
\hypersetup{
 pdfauthor={Lewis Collum},
 pdftitle={EE331 Fall 2019 HW \jobname},
 pdfkeywords={},
 pdfsubject={},
 pdfcreator={Emacs 26.3 (Org mode 9.1.9)}, 
 pdflang={English}}
\begin{document}

\maketitle

\section*{1}
\label{sec:org8db3dda}
\subsection*{Q1: Find the phasor current \(I_S\)}
\label{sec:orgdbe0e8d}
\begin{minted}[framesep=6pt,fontfamily=courier,fontsize=\footnotesize]{python}
v = {'S': 120}

z = {
    'M': 10 + 4j,
    'S': 0.2 + 0.5j,
    'L': 0.3 + 0.1j,
    'X': 3 + 10j
}

z['LM'] = z['L']+z['M']
z['LMX'] = (z['LM']**-1 + z['X']**-1)**-1
z['equivalent'] = z['LMX'] + z['S']

i = {'S': v['S']/z['equivalent']}
print("\(\\boxed{I_S =", f"{i['S']:.3}", "\si{A}}\)")
\end{minted}
\(\boxed{I_S = (11.9-14.2j) \si{A}}\)

\subsection*{Q2: Find the phasor voltage \(V_X\)}
\label{sec:org4df7a82}
\begin{align*}
  \frac{V_S-V_X}{Z_S} &= I_S \\
  \implies V_X &= V_S - Z_SI_S 
\end{align*}

\begin{minted}[framesep=6pt,fontfamily=courier,fontsize=\footnotesize]{python}
v['X'] = v['S'] - i['S']*z['S']
print("\(\\boxed{V_X =", f"{v['X']:.3}", "\si{V}}\)")
\end{minted}
\(\boxed{V_X = (111-3.12j) \si{V}}\)

\subsection*{Q3: Find active power, reactive power, and apparent power flows at point X}
\label{sec:org5aa619d}
\begin{minted}[framesep=6pt,fontfamily=courier,fontsize=\footnotesize]{python}
p = {'X': v['X']*i['S']}
p['apparent'] = abs(p['X'])
p['active'] = p['X'].real
p['reactive'] = p['X'].imag
print(f"Active Power @ X = {p['active']:.5} W")
print(f"Reactive Power @ X = {p['reactive']:.5} VAR")
print(f"Apparent Power @ X = {p['apparent']:.5} VA")
\end{minted}

\begin{verbatim}
Active Power @ X = 1272.7 W
Reactive Power @ X = -1604.0 VAR
Apparent Power @ X = 2047.5 VA
\end{verbatim}

\section*{2}
\label{sec:org745b923}
\begin{minted}[framesep=6pt,fontfamily=courier,fontsize=\footnotesize]{python}
import numpy
import pint
unit = pint.UnitRegistry()

w = 60.0 * 2*numpy.pi * unit.rad/unit.s
v = {'S': 120 * unit.volts, 'L': 110*numpy.exp(1j * numpy.radians(5)) * unit.volts}
l = {'S': 0.1 * unit.mH, 'L': 0.2 * unit.mH}
r = {'S': 0.3 * unit.ohm, 'L': 0.5 * unit.ohm}
z = {'S': r['S'] + w*l['S']*1j, 'L': r['L'] + w*l['L']*1j}
\end{minted}

\subsection*{Q1: Find the phasor current \(I_S\)}
\label{sec:orga1d49e2}
\begin{equation*}
  I_S = \frac{V_S - V_L}{Z_{equivalent}}
\end{equation*}
\begin{minted}[framesep=6pt,fontfamily=courier,fontsize=\footnotesize]{python}
z['equivalent'] = z['S'] + z['L']
i = {'S': (v['S']-v['L'])/z['equivalent']}
print("\(\\boxed{I_S =", f"{i['S'].magnitude:.3}", "\si{A}}\)")
\end{minted}
\(\boxed{I_S = (11.1-13.6j) \si{A}}\)

\subsection*{Q2: Find active power, reactive power, and apparent power flows at load \(V_L\)}
\label{sec:orgc4eada7}
\begin{equation*}
  P_L = I_SV_L
\end{equation*}
\begin{minted}[framesep=6pt,fontfamily=courier,fontsize=\footnotesize]{python}
p = {'L': i['S']*v['L']}
p['active'] = p['L'].real
p['reactive'] = p['L'].imag
p['apparent'] = abs(p['L'])
print(f"Active Power @ X = {p['active'].magnitude:.5} W")
print(f"Reactive Power @ X = {p['reactive'].magnitude:.5} VAR")
print(f"Apparent Power @ X = {p['apparent'].magnitude:.5} VA")
\end{minted}

\begin{verbatim}
Active Power @ X = 1347.1 W
Reactive Power @ X = -1378.8 VAR
Apparent Power @ X = 1927.6 VA
\end{verbatim}

\subsection*{Q3: Monthly cost for operating load. Assume that one month has 30 days.}
\label{sec:orgbce9cb7}
\begin{equation*}
  C_m = \frac{20 \ \si{cents}}{\si{kWh}} \cdot \frac{30\cdot 24 \ \si{h}}{\si{month}} \cdot P_{L\_active}\times 10^{-3} \si{kWh}
\end{equation*}
\begin{minted}[framesep=6pt,fontfamily=courier,fontsize=\footnotesize]{python}
centsPerkWh = 20
daysPerMonth = 30
hoursPerDay = 24
hoursPerMonth = daysPerMonth * hoursPerDay
monthlyDollars = centsPerkWh * hoursPerMonth * p['active'].magnitude /10**3 / 100
print(f"Cost (dollars): {monthlyDollars:.5}")
\end{minted}

\begin{verbatim}
Cost (dollars): 193.98
\end{verbatim}
\end{document}
